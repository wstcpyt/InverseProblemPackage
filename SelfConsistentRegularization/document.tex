%%This is a very basic article template.
%%There is just one section and two subsections.
\documentclass{article}

\usepackage{cite}
\usepackage{graphicx}
\usepackage{caption}
\usepackage{subcaption}
\usepackage[a4paper, total={6in, 8in}]{geometry}

\begin{document}
\title{Self Consistent ill-posed Inverse Pronlem With Cross Validation}
\maketitle

\section{Introduction and Motivation}
The mathematical term well-posed problem stems from a definition given by
Jacques Hdamard. He believed that mathematical models of physical phenomena
should have the properties that.
\begin{enumerate} 
  \item A solution exists.
  \item The solution is unique.
  \item The solution's behavior changes continuously with the initial
  conditions.
\end{enumerate}
Problems that are not well-posed in the sense
of Hadamard are termed ill-posed. Inverse problems are often ill-posed.

Continuum models must often be discretized in order to obtain a numerical
solution, while solutions may be continuous with respect to the initial
conditions, they may suffer from numerical instability when solved with finite
precision, or with erros in the data. Even if a problem is well posed, it may
still be ill-conditioned, meaning that a small error in the intial data can
result in much larger errors in the answers.

In order to solve the ill-posed problem, regularization method is introduced. A
simple form of regularization applied to integral equations, generally termed
Tikhonov regularization, is essentially a trade-off between fitting the data and
reducing a norm of the solution. 

Ill-posed inverse problem with the regularization method can give out a very
good reconstructed result compared to the exact result, but only if the measured
value and the model is accurate enough.

In order to solve the ill-posed inverse problem which the measure value or the 
model is not accurate enougth to give out a good reconstructed result, we
introduce the self consistent regularization method With Cross Validation.

\section{Regularization Methods at Work: A Model Problem from Geophysics}
We use a very simple model to illustrate how we use regularization method to
solve the ill-posed inverse problem. We use a simplified problem from gravity
surveying. An unknown mass distribution with density \(f(t)\) is located at
depth de below the surface. From 0 to 1 on the \(t\) axis shown in
Figure. \ref{fig:GeophysicsIllustrator}. We assume there is no mass outside
this source, which produce a gravity field everywhere. At the surface, along the
s axis in Figure. \ref{fig:GeophysicsIllustrator} from 0 to 1, we mreasure the
vertical component of the gravity field, which reger to as g(s).
 
 \begin{figure}[h!]
  \centering
    \includegraphics[width=0.7\textwidth]{images/GeophysicsIllustrator/illustrator}
  \caption{The geometry of the gravity surveying model problem: \(f(t)\) is the
  mass density at \(t\). and g(s) is the vertical component of the gravity
  field at s.}
  \label{fig:GeophysicsIllustrator}
\end{figure}
The two functions f and g are related via a Fredholm integral equation of the
first kind. The gravity field from an infinitesmally small part of f(t), of
length dt, on the axis is identical to the field from a point mass at t of
strength f(t)dt. Hence, the magnitude of the gravity field along s is
\(f(t)dt/{r^{2x}}\), where \(r = \sqrt {{d^2} + {{(s - t)}^2}} \) is the
distance between the source point at t and the field point at s. The direction
of the gravity field is from the field point to  he source point, and therefore
the measured value of g(s) is
\[dg = \frac{{\sin \theta }}{{{r^2}}}f(t)dt\]
where \(\theta \) is the angle shwon in Figure. \ref{fig:GeophysicsIllustrator}.
Using that \(\sin \theta  = d/r\), we obtain
\[\frac{{\sin \theta }}{{{r^2}}}f(t)dt = \frac{d}{{{{({d^2} + {{(s -
t)}^2})}^{3/2}}}}f(t)dt\]
The total value of g(s) for any \(0 \le s \le 1\) consists of contributions from
all mass along the t axis (from 0 to 1). and it is therefore given by the
integral
\[g(s) = \int\limits_0^1 {\frac{d}{{{{({d^2} + {{(s - t)}^2})}^{3/2}}}}f(t)dt}
\]
This is the forward problem and writing it as
\begin{equation}
\int\limits_0^1 {K(s,t)f(t)dt = g(s),\begin{array}{*{20}{c}}
{0 \le s \le 1}&{}
\end{array}}
\label{eq:integralequation}
\end{equation}
where the function K, which represents the model, is given by
\begin{equation}
k(s,t) = \frac{d}{{{{({d^2} + {{(s - t)}^2})}^{3/2}}}}
\end{equation}
and the righ-hand side g is what we are able to measure. The function K is the
vertical component of the gravity field, measured at s, from a unit point source
locate at t. From K and g we want to compute f, and this is the inverse problem.

Figure.\ref{fig:geoexact} shows an example of the computation of the measured
signal g(s), given the mass distribution f and three different values of the
depth d.

\begin{figure}
	\centering
		\begin{subfigure}[b]{0.7\textwidth}
			\includegraphics[width=\textwidth]{images/FexactPlot/fexactplot}
			\caption{f(t)}
			\label{fig:f(t)}
		\end{subfigure}
		\begin{subfigure}[b]{0.7\textwidth}
			\includegraphics[width=\textwidth]{images/GexactPlot/gexactplot}
			\caption{g(s)}
			\label{fig:g(s)}
		\end{subfigure}
		\caption{(a) shows the function f(the mass density
		distribution), and (b) shows the measured signal g (the
		gravity field) for three different values of the depth d in Figure.
		\ref{fig:GeophysicsIllustrator}}\label{fig:geoexact}
\end{figure}
\subsection{Discretization of Liner Inverse Problems: Quadrature Methods}
We compute approximations \({\tilde f_j}\) to the solution f solely at selected
abscissas \({t_1},{t_2},...{t_n},i.e.\),
\[{\tilde f_j} = \tilde f({t_j}),\begin{array}{*{20}{c}}
{}&{j = 1,2,...,n.}
\end{array}\]

Quadrature methods--also called Nystrom methods--take their basis in the general
quadrature rule of the form 
\[\int\limits_0^1 {\varphi (t)dt = \sum\limits_{j = 1}^n {{\omega _j}\varphi
({t_j})} }  + {E_n}\]
where \(\varphi \) is the function whose integral we want to evaluate, \({E_n}\)
is the quadrature error, \({t_1},{t_2},...,{t_n}\) are the abscissas for the
quadrature rule, and \({\omega _1},{\omega _2},{\omega _3},...,{\omega _n}\) are
the corresponding weights. For example, for the midpoint rule in the interval
\([0, 1]\) we have
\begin{equation}
{t_j} = \frac{{j - \frac{1}{2}}}{n},\begin{array}{*{20}{c}}
{}&{{\omega _j} = \frac{1}{n},\begin{array}{*{20}{c}}
{}&{j = 1,2,...n.}
\end{array}}
\end{array}
\end{equation}
Apply the quadrature methods on Eq.\ref{eq:integralequation}, we arrive at the
relations
\begin{equation}
\sum\limits_{j = 1}^n {{\omega _j}K({s_j},{t_j})} {\tilde f_j} = g({s_i}),\begin{array}{*{20}{c}}
{}&{i = 1,...n.}
\end{array}
\end{equation}

The realtion in Eq.\ref{eq:disintegral} are just a linear system, which can be
also written as
\[\left( {\begin{array}{*{20}{c}}
{{\omega _1}K({s_1},{t_1})}&{{\omega _2}K({s_1},{t_2})}&{...}&{{\omega _n}K({s_1},{t_n})}\\
{{\omega _1}K({s_2},{t_1})}&{{\omega _2}K({s_2},{t_2})}&{...}&{{\omega _1}K({s_2},{t_n})}\\
 \vdots & \vdots &{}& \vdots \\
{{\omega _1}K({s_n},{t_1})}&{{\omega _1}K({s_n},{t_2})}&{...}&{{\omega _1}K({s_n},{t_n})}
\end{array}} \right)\]

or simply \(Ax=b\), where A is an \(n \times n\) matrix. The elements of the
matrix A, the right-hand side b, and the solution vector x are given by
\begin{equation}
\left. {\begin{array}{*{20}{c}}
{{a_{ij}} = {\omega _j}K({s_i},{t_j})}\\
{{x_j} = \tilde f({t_j})}\\
{{b_i} = g({s_i})}
\end{array}} \right\}\begin{array}{*{20}{c}}
{}&{i,j = 1,...,n.}
\end{array}
\end{equation}

\section{Generalized Cross Validation}
The fitness is calculated as:
\begin{equation}
	\mathop {\min }\limits_\lambda  ||A{x_\lambda } - {b^{exact}}||_2^2
\end{equation}
However, we can't calculate it since \(b^{exact}\) is not available. Generalized 
Cross Validation is a classical statistical technique that comes into good use
here \cite{hansen2010discrete}.

Using Generalized Cross Validation(GSV), the fitness can be calculated as:
\begin{equation}
	\mathop {\min }\limits_\lambda  \frac{{||A{x_\lambda } - b||_2^2}}{{{{(m -
	\sum\nolimits_{i = 1}^n {\varphi _i^{[\lambda ]}} )}^2}}}
	\label{eq:disintegral}
\end{equation}


\section{Cellular Evolutionary Algorithm}
Usually EAs assume that the strucutre of the population is panmictic, which
means that any inidividual may interact with any other individual in the
population.
However, this need not be always the case: we often see population sin the
biological and social world in which individuals only interact with a subset of
the rest of the population. This situation can usefully be depicated by using
the concept of a population graph \cite{hoekstra2010simulating}.

\section{Result}
\begin{figure}[h!]
  \centering
    \includegraphics[width=0.7\textwidth]{images/exactmassdensity}
  \caption{Exact f function (mass density distribution)}
  \label{fig:MASS}
\end{figure}
\begin{figure}[h!]
  \centering
    \includegraphics[width=0.7\textwidth]{images/exactgravityfield}
  \caption{Exact signal g (the gravity field)}
  \label{fig:G}
\end{figure}
\begin{figure}[h!]
  \centering
    \includegraphics[width=0.7\textwidth]{images/reconstructedmassdensitygood}
  \caption{The Reconstructed function f(the mass density distribution) after
  self-consistent genetic algorithm, at \(\lambda  = {10^{ - 12}}\), GSV value
  = 1.65e-25}
  \label{fig:RMASSGOOD}
\end{figure}
\begin{figure}[h!]
  \centering
    \includegraphics[width=0.7\textwidth]{images/reconstructedmassdensitybad}
  \caption{The Reconstructed function f(the mass density distribution) without
  self-consistent genetic algorithm, at \(\lambda  = {10^{ - 12}}\), GSV value
  = 7.22e-10}
  \label{fig:RMASSBAD}
\end{figure}

\bibliography{mybib}{}
\bibliographystyle{plain}
\end{document}
